% \iffalse meta-comment
% 
% This is file `subcaption.dtx'.
% 
% $Id: subcaption.dtx 161 2016-05-22 14:39:14Z sommerfeldt $
% $HeadURL: svn+ssh://sommerfeldt@svn.code.sf.net/p/latex-caption/code/branches/3.3/source/subcaption.dtx $
%
% Copyright (C) 2007-2016 Axel Sommerfeldt (axel.sommerfeldt@f-m.fm)
% 
% --------------------------------------------------------------------------
% 
% This work may be distributed and/or modified under the
% conditions of the LaTeX Project Public License, either version 1.3
% of this license or (at your option) any later version.
% The latest version of this license is in
%   http://www.latex-project.org/lppl.txt
% and version 1.3 or later is part of all distributions of LaTeX
% version 2003/12/01 or later.
% 
% This work has the LPPL maintenance status "maintained".
% 
% This Current Maintainer of this work is Axel Sommerfeldt.
% 
% This work consists of the files caption.ins, caption.dtx, caption2.dtx,
% caption3.dtx, bicaption.dtx, ltcaption.dtx, subcaption.dtx, and newfloat.dtx,
% the derived files caption.sty, caption2.sty, caption3.sty,
% bicaption.sty, ltcaption.sty, subcaption.sty, and newfloat.sty,
% and the user manuals caption-deu.tex, caption-eng.tex, and caption-rus.tex.
% 
% \fi
% \CheckSum{165}
%
% \iffalse
%<*driver>
\NeedsTeXFormat{LaTeX2e}[1994/12/01]
\ProvidesFile{subcaption.drv}[2013/04/16 v1.1 Adds a sub-caption feature to the caption package]
\hbadness=9999 \newcount\hbadness \hfuzz=74pt % Make TeX shut up.
%\errorcontextlines=3
%
%\documentclass{ltxdoc}
\setlength\parindent{0pt}
\setlength\parskip{\smallskipamount}
%
\newcommand\LineBreak{\linebreak[3]}
\newcommand\PageBreak{\pagebreak[3]}
\usepackage{ifpdf}
\ifpdf
  \usepackage{mathptmx,courier}
  \usepackage[scaled=0.90]{helvet}
  \addtolength\marginparwidth{15pt}
  \ifdim\paperheight=297mm % a4paper
    \renewcommand\LineBreak{\\}
   \renewcommand\PageBreak{\clearpage}
  \fi
\fi
%
\usepackage[bottom]{footmisc}
\usepackage{array,graphicx,overpic,pict2e,diagbox}
%
\PassOptionsToPackage{breaklinks=true}{hyperref}
\usepackage{hypdoc}
\ifpdf\usepackage{hypdestopt}\fi
\hypersetup{pdfkeywords={LaTeX, package, subcaption},pdfstartpage={},pdfstartview={}}
%
\usepackage{subcaption}[2011/08/01] % needs v1.1 or newer
\DeclareCaptionSubType*[arabic]{table}
\captionsetup[subtable]{labelformat=simple,labelsep=colon}
%
\newcommand*\purerm[1]{{\upshape\mdseries\rmfamily #1}}
\newcommand*\puresf[1]{{\upshape\mdseries\sffamily #1}}
\newcommand*\purett[1]{{\upshape\mdseries\ttfamily #1}}
\let\package\puresf
\let\env\purett \let\opt\purett
%
\newcommand*\csmarg[1]{\texttt{\char`\{#1\char`\}}}
\newcommand*\csoarg[1]{\texttt{\char`\[#1\char`\]}}
\newcommand*\version[2][]{$v#2$}
%
\usepackage{marvosym}
\makeatletter
\newcommand*\INFO{\@ifstar{\@INFO{}}{\@INFO{\vbox to \ht\strutbox}}}
\newcommand*\@INFO[1]{\MARGINSYM{#1{\LARGE\Info}}}
\makeatother
%
\usepackage[alpine]{ifsym}
\newenvironment{background}{\par\bigskip\csname background*\endcsname}{\csname endbackground*\endcsname}
\newenvironment{background*}{\small\MARGINSYM{\Mountain}\ignorespaces}{\par}
%
\newcommand*\MARGINSYM[1]{\hskip 1sp \marginpar{\raggedleft\textcolor{blue}{{#1}}}}
%\newcommand*\NEW[2]{}%\hskip 1sp \marginpar{\footnotesize\sffamily\raggedleft#1\\#2}}
%
\begin{document}
  \DocInput{subcaption.dtx}
\end{document}
%</driver>
% \fi
%
% \let\subsectionautorefname\sectionautorefname
% \let\subsubsectionautorefname\sectionautorefname
%
% \def\thispackage{the \package{subcaption} package}
% \def\Thispackage{The \package{subcaption} package}
%
% \newcommand\NEWfeature{\NEW{New feature}}
% \newcommand\NEWdescription{\NEW{New description}}
%
% \makeatletter
% \newcommand*\Ref{\@ifstar{\@Ref\ref}{\@Ref\autoref}}
% \newcommand*\@Ref[2]{#1{#2}: \textit{\nameref{#2}}}
% \makeatother
% \newcommand*\See[1]{\nopagebreak{\small (See #1)}}
%
% \GetFileInfo{subcaption.drv}
% \let\docdate\filedate
% \GetFileInfo{subcaption.sty}
%
% \title{\texorpdfstring{\Thispackage\thanks{%^^A
%          This package has version number \fileversion, last revised \filedate.}}%^^A
%        {The subcaption package}}
% \author{Axel Sommerfeldt\\
%         \url{http://sourceforge.net/projects/latex-caption/}}
% \date{\docdate}
% \maketitle
% 
% \begin{abstract}
% This package supports typesetting of sub-captions
% (by using the the sub-caption feature of the \package{caption} package).
% \end{abstract}
%
% \begin{background}
% At the end of each section, text marked with the mountain symbol
% will contain background knowledge on how the particular command or
% environment is actually implemented.
% If you just want to use this package as it is, you don't have to read or
% understand them.
% \end{background}
%
% \begin{background*}
% This package demonstrates the usage of |\Declare|\-|Caption|\-|Sub|\-|Type|,
% |\caption|\-|setup{sub|\-|type}|, and the internal hook
% |\caption@sub|\-|type|\-|hook| (offered by the \package{caption} package).
% \end{background*}
% 
% \bigskip
% \INFO*
% \emph{Please note:}
% This package is incompatible with the \package{subfigure} and \package{subfig}
% packages.
%
% \clearpage
% \setcounter{tocdepth}{2}
% \tableofcontents
%
% \clearpage
% \section{Loading the package}
%
% Load this package using
% \begin{quote}
%   |\usepackage|\oarg{options}|{subcaption}|\quad.
% \end{quote}
% The options for \thispackage\ are the same ones as for the \package{caption}
% package, but specify settings which are used for sub-captions
% \emph{additionally}.
% In fact
% \begin{quote}
%   |\usepackage|\oarg{options}|{subcaption}|
% \end{quote}
% is identical to
% \begin{quote}
%   |\usepackage{subcaption}|\\
%   |\captionsetup[sub]|\marg{options}\quad.
% \end{quote}
%
% \bigskip
%
% The default settings for |sub|captions are:
% \begin{quote}
%   |margin=0pt,font+=small,labelformat=parens,labelsep=space,|\\
%   |skip=6pt,list=false,hypcap=false|~\footnote{%^^A
%     This means that sub-captions are not listed in the List of Figures
%     or Tables by default, but you can enable that by
%     specifying the option \texttt{list=true}.}
% \end{quote}
%
% Options specified with |\usepackage[|\ldots|]{sub|\-|caption}| and
% |\caption|\-|setup[sub]{|\ldots|}| will override the ones specified by
% |\caption|\-|setup{|\ldots|}| and |\caption|\-|setup[fig|\-|ure]{|\ldots|}|,
% but are again overwritten by |\caption|\-|setup[sub|\-|figure]{|\ldots|}|
% (same for `table'). So finally we have the following order how
% settings for sub-captions are applied:
% \begin{enumerate}
% \item Global settings
%  {\small(|\usepackage[|\ldots|]{caption}| and |\captionsetup{|\ldots|}|)}
% \item Environmental settings
%  {\small(|\captionsetup[figure|\emph{ -or- }|table]{|\ldots|}|)}
% \item Local settings
%  {\small(|\captionsetup{|\ldots|}| inside |figure| or |table| environment)}
% \item Default `sub' settings
%  {\small(|margin=0pt,font+=small,|\ldots, see above)}
% \item Custom `sub' settings
%  {\small(|\usepackage[|\ldots|]{subcaption}| and |\captionsetup[sub]{|\ldots|}|)}
% \item Environmental `sub' settings
%  {\small(|\captionsetup[subfigure|\emph{ -or- }|subtable]{|\ldots|}|)}
% \item Local `sub' settings
%  {\small(|\captionsetup{|\ldots|}| inside |sub|\-|figure| or |sub|\-|table|)}
% \end{enumerate}
% An example:
% \begin{quote}
%   |\usepackage[labelsep=quad,indention=10pt]{caption}|\\
%   |\usepackage[labelfont=bf,list=true]{subcaption}|\\
%   |\captionsetup[table]{textfont=it,position=top}|\\
%   |\captionsetup[subtable]{textfont=sf}|
% \end{quote}
% causes the captions inside |sub|\-|table| environments to be typeset with
% the settings
% \begin{quote}
%   |indention=10pt,position=top,margin=0pt,font=small,|\\
%   |labelformat=parens,labelsep=space,skip=6pt,hypcap=false,|\\
%   |labelfont=bf,list=true,textfont=sf|\quad.
% \end{quote}
%
% \PageBreak
% \section{The \cs{subcaption} command}
%
% \DescribeMacro\subcaption
% The easiest and most flexible method to apply a sub-caption is by using the
% |\subcaption| command. Its syntax is analogous to the one of the |\caption|
% command and shares its features:
% \begin{quote}
%   |\subcaption|\oarg{list entry}\marg{heading}\\
%   |\subcaption*|\marg{heading}
% \end{quote}
% Please note that the |\subcaption| command \emph{must} be applied inside
% its own box or environment.
%
% An example:
% \begin{quote}
%   |\begin{figure}|\\
%   |  \begin{minipage}[b]{.5\linewidth}|\\
%   |    \centering\large A|\\
%   |    \subcaption{A subfigure}\label{fig:1a}|\\
%   |  \end{minipage}%|\\
%   |  \begin{minipage}[b]{.5\linewidth}|\\
%   |    \centering\large B|\\
%   |    \subcaption{Another subfigure}\label{fig:1b}|\\
%   |  \end{minipage}|\\
%   |  \caption{A figure}\label{fig:1}|\\
%   |\end{figure}|
% \end{quote}
% gives the result:
% \par\bigskip
% \noindent\begin{minipage}{\linewidth}
%   \setcaptiontype{figure}
%   \begin{minipage}[b]{.5\linewidth}
%     \centering\large A
%     \subcaption{A subfigure}\label{fig:1a}
%   \end{minipage}%^^A
%   \begin{minipage}[b]{.5\linewidth}
%     \centering\large B
%     \subcaption{Another subfigure}\label{fig:1b}
%   \end{minipage}
%   \caption{A figure}\label{fig:1}
% \end{minipage}
%
% \bigskip
%
% \begin{background}
% Prepared with |\Declare|\-|Caption|\-|Sub|\-|Type| (offered by the
% \package{caption} package), the caption package command |\set|\-|caption|\-|sub|\-|type|
% becames available.
% Analogous to the |\set|\-|caption|\-|type| command of the \package{caption} package,
% the |\set|\-|caption|\-|sub|\-|type| command sets the sub-type of the box or environment
% (so |\caption| will typeset a sub-caption instead of an ordinary one),
% places a proper hyperlink anchor (non-starred variant only),
% executes options associated with the sub-type etc.\par
% The |\subcaption| command is just a simple combination of
% |\set|\-|caption|\-|sub|\-|type*| and |\caption|.
% \end{background}
%
% \PageBreak
% \section{The subfigure \& subtable environments}
%
% \DescribeEnv{subfigure}
% \DescribeEnv{subtable}
% After loading \thispackage\ the new environments |sub|\-|figure| and
% |sub|\-|table| are available, which have the same (optional \& mandatory)
% arguments as the |mini|\-|page| environment:
% \begin{quote}
%   |\begin{subfigure}|\oarg{pos}\marg{width}\\
%   \ldots\\
%   |\end{subfigure}|
% \end{quote}
% and
% \begin{quote}
%   |\begin{subtable}|\oarg{pos}\marg{width}\\
%   \ldots\\
%   |\end{subtable}|
% \end{quote}
% Inside these environments you use the ordinary |\caption| command
% for typesetting captions. So this example is the same as the last one,
% but uses the |sub|\-|figure| environment:
% \begin{quote}
%   |\begin{figure}|\\
%   |  \begin{|\textcolor{blue}{\texttt{subfigure}}|}[b]{.5\linewidth}|\\
%   |    \centering\large A|\\
%   |    |\textcolor{blue}{\cs{caption}}|{A subfigure}\label{fig:1a}|\\
%   |  \end{|\textcolor{blue}{\texttt{subfigure}}|}%|\\
%   |  \begin{|\textcolor{blue}{\texttt{subfigure}}|}[b]{.5\linewidth}|\\
%   |    \centering\large B|\\
%   |    |\textcolor{blue}{\cs{caption}}|{Another subfigure}\label{fig:1b}|\\
%   |  \end{|\textcolor{blue}{\texttt{subfigure}}|}|\\
%   |  \caption{A figure}\label{fig:1}|\\
%   |\end{figure}|
% \end{quote}
% Using the |sub|\-|figure| or |sub|\-|table| environment instead of |\subcaption|
% has two advantages:
% \begin{itemize}
% \item You can override the setttings for a specific subcaption
% with a |\caption|\-|setup| inside the |sub|\-|figure| or |sub|\-|table| environment, e.g.:
% \begin{quote}
%   |\begin{subfigure}[b]{.5\linewidth}|\\
%   |  \centering\large A|\\
%   |  |\textcolor{blue}{\cs{captionsetup}\csmarg{skip=3pt}}\\
%   |  \caption{A subfigure}\label{fig:1a}|\\
%   |\end{subfigure}|\\
% \end{quote}
% \item Hyperlinks targeted to this subfigure will jump to the beginning
% of the subfigure, and not to the caption of the subfigure
% (if |hypcap=true| is set for sub-captions).
% \See{\Ref{hypcap}}
% \end{itemize}
%
% \bigskip
%
% \begin{background}
% The |subfigure| \& |subtable| environments are just simple minipage
% environments with |\set|\-|caption|\-|sub|\-|type| as first contents line.
% These environments are defined with the help of
% |\caption@For{subtypelist}|, which executes code for every sub-type
% declared with |\Declare|\-|Caption|\-|Sub|\-|Type|.
% \end{background}
%
% \PageBreak
% \section{The \cs{subcaptionbox} command}
%
% \DescribeMacro\subcaptionbox
% A different way of setting sub-figures is
% offered by the |\sub|\-|caption|\-|box| command, which automatically
% aligns the sub-figures resp.~sub-tables by their very first caption line.
%
% Its syntax is:
% \begingroup
% \leftmargini=12pt
% \begin{quote}
%   |\subcaptionbox|\oarg{list entry}\marg{heading}\oarg{width}\oarg{inner-pos}\marg{contents}\\
%   |\subcaptionbox*|\marg{heading}\oarg{width}\oarg{inner-pos}\marg{contents}
% \end{quote}
% \endgroup
% \begin{small}
% The arguments \meta{list entry} \& \meta{heading} will be used for
% typesetting the |\caption|.\par
% \meta{width} is the width of the resulting |\par|\-|box|; the default value is
% the width of the contents.\par
% \meta{inner-pos} specifies how the contents will be justified inside the
% resulting |\parbox|;
% it can be either `c' (for |\centering|), `l' (for |\ragged|\-|right|),
% `r' (for |\ragged|\-|left|), or `s' (for no special justification).
% The default is `c'.
% (But you can use any justification defined with
% |\Declare|\-|Caption|\-|Jus|\-|ti|\-|fi|\-|ca|\-|tion| as well,
% e.g.~`|center|\-|last|'.)
% \end{small}
%
% \bigskip
%
% Using |\sub|\-|caption|\-|box|, the baseline of the resulting box will be placed
% right between contents and heading, so usually you don't have to care about the
% vertical alignment of the sub-figures for yourself.
% Also the hyperlink anchor is placed properly with respect to the |hyp|\-|cap=|
% setting.
%
% One example:
% \begin{quote}
%   |\begin{figure}|\\
%   |  \centering|\\
%   |  \subcaptionbox{A cat\label{cat}}|\\
%   |    {\includegraphics{cat}}|\\
%   |  \subcaptionbox{An elephant\label{elephant}}|\\
%   |    {\includegraphics{elephant}}|\\
%   |  \caption{Two animals}\label{animals}|\\
%   |\end{figure}|
% \end{quote}
% gives the result:\par
% \noindent\begin{minipage}{\linewidth}
%   \setcaptiontype{figure}
%   \centering
%   \subcaptionbox{A cat\label{cat}}
%     {\includegraphics[width=30pt]{cat}}
%   \subcaptionbox{An elephant\label{elephant}}
%     {\includegraphics[width=.4\textwidth]{elephant}}
%   \caption[Two animals]{Two animals~\footnotemark}\label{animals}
% \end{minipage}
% \footnotetext{The pictures were taken with permission from the
%   \LaTeX\ Companion\cite{TLC2} examples.}
%
% \bigskip
%
% As you see the result is not satisfying;
% the caption below the cat looks ugly because of the small width of the
% graphic. This can be solved by using the optional arguments of
% |\sub|\-|caption|\-|box|, increasing the width of the resulting box:
% \begin{quote}
%   |  |\ldots\\
%   |  \subcaptionbox{A cat\label{cat}}|\\
%   |    |\textcolor{blue}{\csoarg{2.5cm}}|{\includegraphics{cat}}|\\
%   |  |\ldots
% \end{quote}
% \noindent\begin{minipage}{\linewidth}
%   \setcaptiontype{figure}
%   \centering
%   \subcaptionbox{A cat\label{cat2}}
%     [2.5cm]{\includegraphics[width=30pt]{cat}}
%   \subcaptionbox{An elephant\label{elephant2}}
%     {\includegraphics[width=.4\textwidth]{elephant}}
%   \caption{Two animals}\label{animals2}
% \end{minipage}
%
% \bigskip
%
% Furthermore the main caption, which is centered with respect to the
% |\text|\-|width|, looks mis-aligned with respect to the sub-captions.
% This can (again) be solved by using the optional arguments of
% |\sub|\-|caption|\-|box|, giving both boxes the same width, for example:
% \begin{quote}
%   |  |\ldots\\
%   |  \subcaptionbox{A cat\label{cat}}|\\
%   |    |\textcolor{blue}{\csoarg{.4\cs{linewidth}}}|{\includegraphics{cat}}%|\\
%   |  \subcaptionbox{An elephant\label{elephant}}|\\
%   |    |\textcolor{blue}{\csoarg{.4\cs{linewidth}}}|{\includegraphics{elephant}}|\\
%   |  |\ldots
% \end{quote}
% \noindent\begin{minipage}{\linewidth}
%   \setcaptiontype{figure}
%   \centering
%   \subcaptionbox{A cat\label{cat3}}
%     [.4\linewidth]{\includegraphics[width=39.34724pt]{cat}}%^^A
%   \subcaptionbox{An elephant\label{elephant3}}
%     [.4\linewidth]{\includegraphics[width=.4\textwidth]{elephant}}
%   \caption{Two animals}\label{animals3}
% \end{minipage}
%
% \bigskip
%
% \iffalse
% \noindent\begin{minipage}{\linewidth}
%   \setcaptiontype{figure}
%   \centering
%   \hbox{\subcaptionbox{An elephant\label{elephant4}}
%     {\includegraphics[width=.4\textwidth]{elephant}}%^^A
%   \vbox{\subcaptionbox{Cat 1\label{cat4.1}}
%     {\includegraphics[width=30pt]{cat}}\par
%   \subcaptionbox{Cat 2\label{cat4.2}}
%     {\includegraphics[width=30pt]{cat}}}}%^^A
%   \caption{Two animals}\label{animals4}
% \end{minipage}
% \bigskip
% \fi
%
% \begin{background}
% The |\sub|\-|caption|\-|box| is a |\par|\-|box| with
% |\set|\-|caption|\-|sub|\-|type| as first contents line.
% \iffalse See implementation for details.\fi
% \end{background}
%
% \PageBreak
% \section{The \cs{DeclareCaptionSubType} command}
%
% \DescribeMacro\DeclareCaptionSubType
% For using the sub-caption feature of the \package{caption} package some
% commands and counters must be prepared. This is done with
% \iffalse\footnote{%^^A
% \cs{newsubfloat} offered by the \package{subfig} package\cite{subfig}
% could be used for this purpose as well.}\fi
% \begin{quote}
%  |\DeclareCaptionSubType|\oarg{numbering scheme}\marg{type}\\
%  |\DeclareCaptionSubType*|\oarg{numbering scheme}\marg{type}
% \end{quote}
% For the environments |figure| \& |table|, and all the ones
% defined with |\Declare|\-|Floating|\-|Environment| offered by the
% \package{newfloat} package, this will be done automatically,
% but for other environments (e.g.~the ones defined with |\newfloat| offered by the
% \package{float} package or |\Declare|\-|New|\-|Float|\-|Type| offered by the
% \package{floatrow} package) this has to be done manually.
%
% \medskip
%
% The starred variant provides the sub-caption numbering format
% \meta{type}|.|\meta{subtype} (e.g.~`|1.2|') while the non-starred variant
% simply uses \meta{subtype} (e.g.~`|a|').
%
% \begin{small}
% Own numbering formats can be created by redefining |\thesub|\meta{type}, e.g.
% \begin{quote}|\DeclareCaptionSubType*{figure}|\\
% |\renewcommand\thesubfigure{\thefigure\alph{subfigure}}|\end{quote}
% would give you sub-caption numbers like `|1b|'.
% \end{small}
%
% The default numbering scheme is |alph|, but you can use any \LaTeX\ (or self-defined)
% command name here which converts a counter to a text value, e.g. |arabic|, |roman|,
% |Roman|, |alph|, |Alph|, |fnsymbol|, \ldots
%
% But |\DeclareCaptionSubType| is not only for defining new sub-caption types,
% you can use this command for re-definitions as well, e.g.
% \begingroup
% \leftmargini=12pt
% \begin{quote}
%   |\DeclareCaptionSubType*[arabic]{table}|\\
%   |\captionsetup[subtable]{labelformat=simple,labelsep=colon}|
% \end{quote}
% \endgroup
% \pagebreak[2]
% will give you sub-captions in |table|s like these ones:
% \par\bigskip
% \noindent\begin{minipage}{\linewidth}
%   \setcaptiontype{table}
%   \centering
%   \caption{Two tables}
%   \subcaptionbox{Table one}[3cm][c]{\begin{tabular}{cc}A & B\\ C & D\\ \end{tabular}}
%   \subcaptionbox{Table two}[3cm][c]{\begin{tabular}{cc}E & F\\ G & H\\ \end{tabular}}
% \end{minipage}
%
% \begin{background}
% |\Declare|\-|Caption|\-|Sub|\-|Type| is an integral part of the \package{caption}
% package kernel.
% \end{background}
%
% \PageBreak
% \section{References}
%
% The macro |\the|\-\meta{counter} is not only responsible for the look of the \meta{counter},
% but for the look of the references typeset with |\ref|, too. References will be prefixed by
% \LaTeX{} with the internal macro |\p@|\-\meta{counter}.
%
% |\Declare|\-|Caption|\-|Sub|\-|Type| will define both of them for sub-captions
% (e.g. |sub|\-|figure| and |sub|\-|table|), and as you have seen in the last section
% |\Declare|\-|Caption|\-|Sub|\-|Type| will give you some options to control the
% internal (re-)definition of |\the|\-\meta{counter} and |\p@|\-\meta{counter}.
%
% \DescribeMacro\thesubfigure
% \DescribeMacro\p@subfigure
% For example |\thesubfigure| and |\p@subfigure| are (as default) internally defined as
% \begin{quote}
% |\newcommand\thesubfigure{\alph{subfigure}}|\\
% |\newcommand\p@subfigure{\thefigure}|
% \end{quote}
% so the label of sub-captions will look like `|a|' (decorated by the selected label format),
% while references will look like `|1a|' since they are prefixed by |\p@sub|\-|figure| $=$
% |\the|\-|figure|.
%
% After |\Declare|\-|Caption|\-|Sub|\-|Type*[arabic]{figure}|, |\the|\-|sub|\-|figure| and
% |\p@sub|\-|figure| will look like
% \begin{quote}
% |\renewcommand\thesubfigure{\thefigure.\arabic{subfigure}}|\\
% |\renewcommand\p@subfigure{}|
% \end{quote}
%
% But if you want detailed control on how the references will look like,
% the options of |\Declare|\-|Caption|\-|Sub|\-|Type| are potentially not sufficient.
% In this case one need to redefine these two macros on his/her own.
% Some examples:
%
% If you want parentheses around the sub-figure part of the reference,
% so they will look like `|1(a)|', you may get them this way:
% \begin{quote}
% |\usepackage[labelformat=simple]{subcaption}|\\
% |\renewcommand\thesubfigure{(\alph{subfigure})}|
% \end{quote}
% {\small (\emph{Note:} Since |parens| is the default label format you will get double
% parentheses in sub-captions when not specifiying a different label format, e.g. |simple|.)}
%
% But if you want only a closing parenthesis, so references will look like `|1a)|',
% but the sub-captions itself should still look like `|(a)|',
% this would be a possible solution:
% \begin{quote}
% |\usepackage{subcaption}|\\
% |\renewcommand\thesubfigure{\alph{subfigure})}|\\
% |\DeclareCaptionLabelFormat{opening}{(#2}|\\
% |\captionsetup[subfigure]{labelformat=opening}|
% \end{quote}
%
% {\small(Please note that you need to surround redefinitions of |\p@|\-\meta{counter}
% with |\makeatletter| and |\makeatother|. See
% \url{http://tex.stackexchange.com/questions/8351/}
% for details.)}
%
% \pagebreak[3]
% \subsection{The \cs{subref} command}
%
% While |\ref|\marg{key} (and |\ref*|\marg{key}, if the \package{hyperref}
% package is used) usually gives a combined result representing the main
% caption counter and the sub-caption one, it is sometimes useful to have
% a reference to the sub-caption only. For this purpose you can use
% \begin{quote}
%   |\subref|\marg{key}\\
%   |\subref*|\marg{key}~\footnote{%^^A
%     Like \cs{ref*}, \cs{subref*} is only available if the \package{hyperref}
%     package\cite{hyperref} is used.}%^^A
%     \qquad.
% \end{quote}
% So for example |\ref{cat}| gives the result `\ref{cat}' but |\subref{cat}|
% gives `\subref{cat}'.
%
% \begin{small}
% \emph{Note:} If the sub-caption was (re-)defined with the starred variant
% |\Declare|\-|Caption|\-|Sub|\-|Type*|, both |\ref| and |\sub|\-|ref| usually gives
% the same result.
% \end{small}
%
% \begin{background}
% The |\sub|\-|ref| command demonstrates the usage of |\caption@sub|\-|type|\-|hook|
% which will be called during |\caption|\-|setup{sub|\-|type}|.
% \end{background}
%
% \pagebreak[3]
% \subsection{The \opt{subrefformat=} option}
%
% \DescribeMacro{subrefformat=}
% By applying |\Declare|\-|Caption|\-|Sub|\-|Type|, or by redefining |\the|\-\meta{counter}
% and |\p@|\-\meta{counter}, you will change the look of references typeset with |\ref|
% \emph{and} |\sub|\-|ref|.
% But maybe you only want to change the output of |\sub|\-|ref| without
% affecting the references typeset with |\ref|?
% This is possible, too, by using the option \opt{subrefformat}:
% \begin{quote}
%   |\captionsetup{subrefformat=|\meta{label format}|}|
% \end{quote}
% This one will choose a label format (either a pre-defined one, or a one defined with
% |\Declare|\-|Caption|\-|Label|\-|Format|) as decorative element to sub-references.
% The default one is |simple| which has no decorative elements but simply typeset
% the reference as it is.
%
% For example
% \begin{quote}
%   |\captionsetup{subrefformat=parens}|
% \end{quote}
% will result in references (typeset with |\ref|) like `|1a|' but sub-references
% (typeset with |\subref|) like `|(a)|'.
%
% \subsection{Referencing sub-figures without sub-captions}
%
% \DescribeMacro\phantomsubcaption
% \DescribeMacro\phantomcaption
% If you don't want to give a sub-figure a caption, because the picture itself
% already contains the caption, or for some other reason, you can use the command
% \begin{quote}
%   |\phantomsubcaption|
% \end{quote}
% instead of |\sub|\-|caption|, or
% -- when inside a |sub|\-|figure| or |sub|\-|table| environment --
% |\phantom|\-|caption| instead of |\caption|.
% |\phantom|\-|sub|\-|caption| and |\phantom|\-|caption| do not have any arguments, and
% they do not generate any output, but give you an anchor for a |\label| command
% which can be placed afterwards.
% Furthermore it increases the sub-figure resp. sub-table counter.
%
% Please note that -- just like |\sub|\-|caption| -- the |\phantom|\-|sub|\-|caption|
% command \emph{must} be applied inside its own group, box, or environment.
%
% \pagebreak[3]
% An example:
% \begin{quote}
%   |\begin{figure}|\\
%   |  \centering|\\
%   |  {\includegraphics{cat_with_a}|\\
%   |   \phantomsubcaption\label{cat}}|\\
%   |  {\includegraphics{elephant_with_b}|\\
%   |   \phantomsubcaption\label{elephant}}|\\
%   |  \caption{Two animals: \subref{cat} a huge cat,|\\
%   |           and \subref{elephant} an elephant}|\\
%   |\end{figure}|
% \end{quote}
%
% \noindent\begin{minipage}{\linewidth}
%   \setcaptiontype{figure}
%   \centering
%  {\begin{overpic}[width=60pt]{cat}
%     \put(40,34){(a)}
%   \end{overpic}
%   \phantomsubcaption\label{cat6.3}}
%  {\begin{overpic}[width=.4\textwidth]{elephant}
%     \put(60,50){(b)}
%   \end{overpic}
%   \phantomsubcaption\label{elephant6.3}}
%   \captionsetup{subrefformat=parens}
%   \caption{Two animals: \subref{cat6.3} a huge cat, and \subref{elephant6.3} an elephant}
% \end{minipage}
%
% \pagebreak[3]
% \subsection{Where to place the \cs{label} command?}
% \label{label}
%
% When using |\sub|\-|caption| or |\phantom|\-|sub|\-|caption|,
% or |\caption| or |\phantom|\-|caption| inside a |sub|\-|figure| or |sub|\-|table| environment,
% the |\label| can be either placed inside the caption text or right after the |\sub|\-|caption| or
% |\caption| command, e.g:
% \begin{quote}
%   |\subcaption{Some text here\label{label1}}|\\
%    \ldots\\
%   |\subcaption{Some other text}\label{label2}}|\\
%    \ldots\\
%   |\subcaption{Something different}|\\
%   |\label{label3}|
% \end{quote}
%
% When using the |\sub|\-|caption|\-|box| command, the |\label| should be placed inside
% the caption text, e.g.:
% \begin{quote}
%   |\subcaptionbox{A description here\label{label4}}|\\
%   |              {Some content here}|\\
%    \ldots\\
%   |\subcaptionbox[List-of-Figures entry]|\\
%   |              {A description here\label{label5}}|\\
%   |              {Some content here}|
% \end{quote}
%
% \pagebreak[3]
% \subsection{Where do hyperlinks jump?}
% \label{hypcap}
%
% For the |subfigure| \& |subtable| environments and |\subcaptionbox| boxes
% (and own constructs which use |\set|\-|caption|\-|sub|\-|type|) the
% hyperlink anchors will be placed in respect to the |hypcap=| setting.
% While usage of this option is straight-forward for ordinary captions,
% the usage for sub-captions depends on the setting regarding the main captions.
% This table gives you an overview where the hyperlinks will jump:\par
%
% \bigskip
% \begin{small}
% \centering
% \DeleteShortVerb{\|}
% \renewcommand\arraystretch{1.5}
% \begin{tabular}{|l|p{3cm}p{3cm}|}
% \hline
% \backslashbox{subcaption}{caption} &
%   \multicolumn{1}{c}{\texttt{hypcap=false}} &
%   \multicolumn{1}{c|}{\texttt{hypcap=true}} \\
% \hline
% \raisebox{-1.5ex}[1.5ex]{\texttt{hypcap=false}} &
%   sub-caption & figure or table\par\hfill\textit{(default setting)}\\
% \raisebox{-1.5ex}[1.5ex]{\texttt{hypcap=true}} &
%   sub-figure or\par sub-table & sub-figure or\par sub-table \\
% \hline
% \end{tabular}\par
% \MakeShortVerb{\|}
% \end{small}
% \bigskip
%
% But if |\subcaption| is used and |hypcap=true| is set for sub-captions,
% \thispackage\ does not know where the sub-figure or sub-table actually
% begins, so it will jump to the sub-caption instead.
%
% \medskip
% \emph{Remember:} If you use the \package{hypcap} package\cite{hypcap},
% it controls the placement of the hyperlink anchors, making the rules
% above invalid.
%
% \bigskip
% {\small(See also the documentation of the \package{caption} package,
% sections about \package{hyperref} \& \package{hypcap}.)}
%
% \iffalse
% --------------------------------------------------------------------------- %
% \fi
%
% \pagebreak[3]
% \section{Beyond this package}
% \label{floatrow}
%
% For a more advanced usage of the sub-caption feature of the
% \package{caption} package, please take a look at the \package{floatrow}
% package\cite{floatrow} which provides the powerful \texttt{subfloatrow}
% environment for typesetting sub-figures.
%
% \iffalse
% --------------------------------------------------------------------------- %
% \fi
%
% \pagebreak[3]
% \section{Thanks}
%
% I would like to thank
% Stephen Dalton
% who helped to make this package a better one.
%
% \iffalse
% --------------------------------------------------------------------------- %
% \fi
%
% \StopEventually{%^^A
% \begin{thebibliography}{9}
%   \bibitem{TLC2}
%   Frank Mittelbach and Michel Goossens:\\
%   \newblock {\em The {\LaTeX} Companion (2nd.~Ed.)},
%   \newblock Addison-Wesley, 2004.
%   \bibitem{floatrow}
%   Olga Lapko:\\
%   \href{http://www.ctan.org/pkg/floatrow}%
%        {\emph{The floatrow package documentation}},
%   2007/12/24
%   \bibitem{hyperref}
%   Sebastian Rahtz \& Heiko Oberdiek:\\
%   \href{http://www.ctan.org/pkg/hyperref}%
%        {\emph{Hypertext marks in \LaTeX}},
%   November 12, 2007
%   \bibitem{hypcap}
%   Heiko Oberdiek:\\
%   \href{http://www.ctan.org/pkg/hypcap}%
%        {\emph{The hypcap package -- Adjusting anchors of captions}},
%   2007/04/09
% \iffalse
%   \bibitem{subfig}
%   Steven D. Cochran:\\
%   \href{http://www.ctan.org/pkg/subfig}%
%        {\emph{The subfig package}},
%   2005/07/05
% \fi
% \end{thebibliography}
% }
%
% \iffalse
% --------------------------------------------------------------------------- %
% \fi
%
% \DoNotIndex{\\,\_,\ ,\@@par}
% \DoNotIndex{\@bsphack}
% \DoNotIndex{\@car,\@cdr,\@classoptionslist,\@cons,\@currext,\@currname}
% \DoNotIndex{\@ehc,\@ehd,\@empty,\@esphack,\@expandtwoargs}
% \DoNotIndex{\@for,\@firstofone,\@firstoftwo}
% \DoNotIndex{\@gobble,\@gobblefour,\@gobbletwo,\@hangfrom}
% \DoNotIndex{\@ifnextchar,\@ifpackagelater,\@ifpackageloaded}
% \DoNotIndex{\@ifstar,\@ifundefined,\@latex@error,\@namedef,\@nameuse}
% \DoNotIndex{\@onlypreamble,\@parboxrestore,\@plus,\@ptionlist}
% \DoNotIndex{\@removeelement,\@restorepar,\@secondoftwo,\@setpar}
% \DoNotIndex{\@tempa,\@tempboxa,\@tempdima,\@tempdimb,\@tempdimc,\@tempb,\@tempc}
% \DoNotIndex{\@testopt}
% \DoNotIndex{\@undefined,\@unprocessedoptions,\@unusedoptionlist}
% \DoNotIndex{\p@,\z@}
% \DoNotIndex{\active,\addtocounter,\addtolength,\advance,\aftergroup}
% \DoNotIndex{\baselineskip,\begin,\begingroup,\bfseries,\box}
% \DoNotIndex{\catcode,\centering,\changes,\csname,\def,\divide,\do,\downarrow}
% \DoNotIndex{\edef,\else,\empty,\end,\endcsname,\endgraf,\endgroup,\expandafter}
% \DoNotIndex{\fi,\footnotesize,\global}
% \DoNotIndex{\hangindent,\hbox,\hfil,\hsize,\hskip,\hspace,\hss}
% \DoNotIndex{\ifcase,\ifdim,\ifnum,\ifodd,\ifvoid,\ifvmode}
% \DoNotIndex{\ifx,\ignorespaces,\itshape}
% \DoNotIndex{\Large,\large,\leavevmode,\leftmargini,\leftskip,\let,\linewidth}
% \DoNotIndex{\llap,\long,\m@ne,\margin,\mdseries,\message}
% \DoNotIndex{\newcommand,\newdimen,\newlength,\newline,\newif,\newsavebox}
% \DoNotIndex{\next,\nobreak,\nobreakspace,\noexpand,\noindent,\numberline}
% \DoNotIndex{\normalcolor,\normalfont,\normalsize,\or,\par,\parbox,\parfillskip}
% \DoNotIndex{\parindent,\parskip,\prevdepth,\protect,\protected@edef,\protected@write}
% \DoNotIndex{\providecommand,\quad}
% \DoNotIndex{\raggedleft,\raggedright,\relax,\renewcommand,\RequirePackage}
% \DoNotIndex{\rightskip,\rmfamily}
% \DoNotIndex{\sbox,\scriptsize,\scshape,\setbox,\setlength,\sffamily,\slshape}
% \DoNotIndex{\small,\string,\space,\strut}
% \DoNotIndex{\textheight,\the,\toks@,\typeout,\ttfamily}
% \DoNotIndex{\unvbox,\uparrow,\upshape,\usebox,\usepackage}
% \DoNotIndex{\value,\vbox,\vsize,\vskip,\wd,\width,\z@skip}
% \DoNotIndex{\AtBeginDocument,\AtEndOfPackage,\CurrentOption,\DeclareOption}
% \DoNotIndex{\ExecuteOptions,\GenericWarning,\IfFileExists,\InputIfFileExists}
% \DoNotIndex{\NeedsTeXFormat,\MessageBreak}
% \DoNotIndex{\PackageError,\PackageInfo,\PackageWarning,\PackageWarningNoLine}
% \DoNotIndex{\PassOptionsToPackage,\ProcessOptions,\ProvidesPackage}
%
% \iffalse
% --------------------------------------------------------------------------- %
% \fi
%
% \setlength{\parskip}{0pt plus 1pt}
% \changes{v0.1}{2007/09/01}{First demo}
% \changes{v0.2}{2007/11/11}{\cs{subcaptionbox} added}
% \changes{v0.3}{2007/12/06}{Adapted to \package{caption} package \version{3.1f}}
% \changes{v1.0}{2008/03/16}{\cs{subfloat} added}
% \changes{v1.0}{2010/10/27}{An error message will be issued when the subfigure or subfig package is loaded}
% \changes{v1.0}{2011/01/22}{Undocumented command \cs{subfloat} removed}
%
% \newcommand*\Note[2][Note]{\par{\small\emph{#1:} #2}}
%
% \iffalse
% --------------------------------------------------------------------------- %
% \fi
%
% \clearpage
% \section{The implementation}
% \iffalse
%<*package>
% \fi
%
% \subsection{Identification}
%
%    \begin{macrocode}
\NeedsTeXFormat{LaTeX2e}[1994/12/01]
\def\caption@tempa$Id: #1 #2 #3-#4-#5 #6${%
  \def\caption@tempa{#3/#4/#5 }\def\caption@tempb{#2 }}
\caption@tempa $Id: subcaption.dtx 161 2016-05-22 14:39:14Z sommerfeldt $
\ProvidesPackage{subcaption}[\caption@tempa v1.1-\caption@tempb Sub-captions (AR)]
%    \end{macrocode}
%
% \subsection{Initial code}
%
% Since we base on the \package{caption} package we load it here.
%    \begin{macrocode}
\RequirePackage{caption}[2012/03/25] % needs v3.3 or newer
%    \end{macrocode}
%
% \begin{macro}{\subcaption@CheckCompatibility}
% \changes{v1.1}{2011/09/01}{Compatibility error added}
% \changes{v1.1}{2016/05/22}{The presence of \package{subfigure} or \package{subfig} will be checked \cs{AtBeginDocument}, too}
% Since we are incompatible to them an error message will be issued when
% the \package{subfigure} or \package{subfig} package is loaded.
%    \begin{macrocode}
\newcommand\subcaption@CheckCompatibility{%
%    \end{macrocode}
%    \begin{macrocode}
  \@ifpackageloaded{subfigure}{%
    \PackageError{subcaption}%
      {This package can't be used in cooperation\MessageBreak
       with the subfigure package}%
      {\subcaption@EH}%
    \endinput}{}%
%    \end{macrocode}
%    \begin{macrocode}
  \@ifpackageloaded{subfig}{%
    \PackageError{subcaption}%
      {This package can't be used in cooperation\MessageBreak
       with the subfig package}%
      {\subcaption@EH}%
    \endinput}{}%
%    \end{macrocode}
%    \begin{macrocode}
}
%    \end{macrocode}
%    \begin{macrocode}
\newcommand*\subcaption@EH{%
  If you do not understand this error, please take a closer look\MessageBreak
  at the documentation of the `subcaption' package, especially the\MessageBreak
  section about errors.\MessageBreak\@ehc}
%    \end{macrocode}
%    \begin{macrocode}
\subcaption@CheckCompatibility
%    \end{macrocode}
%    \begin{macrocode}
\caption@AtBeginDocument{%
%    \end{macrocode}
%    \begin{macrocode}
  \caption@ifcompatibility{%
    \caption@Error{%
      The `subcaption' package does not work correctly\MessageBreak
      in compatibility mode}}{}%
%    \end{macrocode}
%    \begin{macrocode}
  \subcaption@CheckCompatibility
  \let\subcaption@CheckCompatibility\@undefined
  \let\subcaption@EH\@undefined
%    \end{macrocode}
%    \begin{macrocode}
}
%    \end{macrocode}
% \end{macro}
%
% \iffalse
% \subsection{Declaration of options}
% We do not have own options.
% \fi
%
% \subsection{Execution of options}
%
% We use |\caption@ExecuteOptions| and |\caption@ProcessOptions| here to add
% the options to the `|sub|' option list instead of executing them immediately.
%    \begin{macrocode}
\caption@SetupOptions{subcaption}{\captionsetup[sub]{#2}}%
\caption@ExecuteOptions{subcaption}{%
  font+=small,labelformat=parens,labelsep=space,skip=6pt,list=0,hypcap=0}
\caption@ProcessOptions*{subcaption}
%    \end{macrocode}
%
% \subsection{Main code}
%
% \changes{v1.1}{2011/10/30}{Adapted to the newfloat package}
% We call |\Declare|\-|Caption|\-|Sub|\-|Type| for |figure|, |table|,
% and each caption type declared with |\Declare|\-|Floating|\-|Environment| here.
%    \begin{macrocode}
\caption@ForEachType{\DeclareCaptionSubType{#1}}
%    \end{macrocode}
%
% \begin{macro}{\newsubfloat}
% \changes{v1.1}{2016/01/31}{Adapted to the memoir document class}
% We re-define |\new|\-|sub|\-|float| (offered by the \package{memoir} document class),
% so our stuff will be used instead.
%    \begin{macrocode}
\caption@ifundefined\newsubfloat{}{%
  \renewcommand*\newsubfloat{\DeclareCaptionSubType}}
%    \end{macrocode}
% \end{macro}
%
% \pagebreak[3]
% \subsubsection{The \cs{subcaption} command}
%
% \begin{macro}{\subcaption}
% \changes{v1.1}{2012/04/06}{\cs{newcommand} changed to \cs{def}
%     so it works with the \package{memoir} document class, too}
% Without a prefacing |\set|\-|caption|\-|sub|\-|type|, |\sub|\-|caption| is some kind
% of |\caption|\-|of{sub|\-|\@cap|\-|type}|.
% \Note{Like \cs{captionof}, this command is designed to be used inside an
% own group!}
%    \begin{macrocode}
\def\subcaption{%
  \caption@iftype
    {\setcaptionsubtype*\caption}%
    {\caption@Error{\noexpand\subcaption outside float}%
     \caption@gobble}}%
%    \end{macrocode}
% But with a prefacing |\set|\-|caption|\-|sub|\-|type|, |\sub|\-|caption| is simply
% |\caption|.
%    \begin{macrocode}
\g@addto@macro\caption@subtypehook{%
  \let\subcaption\caption}
%    \end{macrocode}
% \end{macro}
%
% \pagebreak[3]
% \subsubsection{The \cs{phantomsubcaption} command}
%
% \begin{macro}{\phantomsubcaption}
% \changes{v1.1}{2011/08/17}{This macro added}
% Same as |\phantom|\-|caption|, but for subfigures.
%    \begin{macrocode}
\newcommand*\phantomsubcaption{%
  \caption@iftype
    {\setcaptionsubtype*\phantomcaption}%
    {\caption@Error{\noexpand\phantomsubcaption outside float}}}%
%    \end{macrocode}
%    \begin{macrocode}
\g@addto@macro\caption@subtypehook{%
  \let\phantomsubcaption\phantomcaption}
%    \end{macrocode}
% \end{macro}
%
% \pagebreak[3]
% \subsubsection{The subfigure \& subtable environments}
%
% \begin{macro}{subfigure}
% \begin{macro}{subtable}
% This is just an ordinary \env{minipage} environment with
% |\setcaptionsubtype| as first contents line.
% It will be defined using the helper macro |\caption@For{sub|\-|type|\-|list}|
% offered by the \package{caption} kernel, so for every caption type
% declared with |\Declare|\-|Floating|\-|Environment| a corresponding `sub' environment
% will be defined automatically.
%    \begin{macrocode}
\caption@For{subtypelist}{%
  \newenvironment{sub#1}%
    {\caption@withoptargs\subcaption@minipage}%
    {\endminipage}}%
%    \end{macrocode}
%    \begin{macrocode}
\newcommand*\subcaption@minipage[2]{%
  \minipage#1{#2}%
  \setcaptionsubtype\relax}
%    \end{macrocode}
% \end{macro}
% \end{macro}
%
% \pagebreak[3]
% \subsubsection{The \cs{subcaptionbox} command}
%
% \begin{macro}{\subcaptionbox}
% \changes{v1.0}{2008/05/06}{Adapted to the \opt{rule} option of the \package{caption} package}
% \changes{v1.0}{2008/08/31}{Definition and usage of \cs{subcaption@hrule} added}
% \changes{v1.0}{2010/12/17}{Uses \cs{caption@box} now}
% \changes{v1.1}{2011/08/16}{Adapted to actual version of \cs{caption@box}}
% \changes{v1.1}{2012/04/09}{Adapted to actual version of \cs{caption@ibox}}
% A |\parbox| with contents and sub-caption, separated by an invisible |\hrule|.
%    \begin{macrocode}
\newcommand*\subcaptionbox{%
  \caption@withoptargs{\caption@ibox\setcaptionsubtype}}
%    \end{macrocode}
% \end{macro}
%
% \pagebreak[3]
% \subsubsection{The \cs{subref} command}
%
% At |\captionsetup{subtype}|, we redefine |\label|.
%    \begin{macrocode}
\g@addto@macro\caption@subtypehook{%
  \ifx\label\subcaption@label \else
    \let\subcaption@ORI@label\label
    \let\label\subcaption@label
  \fi}
%    \end{macrocode}
%
% \begin{macro}{\subcaption@label}
% \changes{v1.1}{2011/09/12}{Redefinition of \cs{SK@} added}
% \changes{v1.1}{2011/09/12}{Uses \cs{caption@withoptargs} now}
% \changes{v1.1}{2016/02/20}{Unwanted space removed}
% When a label will be placed for a sub-caption, we automatically place
% a second one for |\subref|, too. This second label will contain
% the sub-type counter only.
%    \begin{macrocode}
\newcommand*\subcaption@label{%
  \caption@withoptargs\subcaption@@label}
%    \end{macrocode}
%    \begin{macrocode}
\newcommand*\subcaption@@label[2]{%
  \@bsphack\begingroup
    \subcaption@ORI@label#1{#2}%
    \let\SK@\@gobbletwo
    \protected@edef\@currentlabel{\csname thesub\@captype\endcsname}%
    \subcaption@ORI@label#1{sub@#2}%
  \endgroup\@esphack}
%    \end{macrocode}
% \end{macro}
%
% \begin{macro}{\subref}
% \changes{v1.1}{2011/08/14}{Caption option \opt{subrefformat=} added}
% \changes{v1.1}{2011/08/18}{Uses \cs{caption@setoptions*} now}
% \changes{v1.1}{2012/01/12}{Usage of \cs{caption@setoptions*} replaced by \cs{caption@setoptions}}
% \changes{v1.1}{2012/04/09}{Revised}
% This one calls |\ref| with the second label. (see |\subcaption@label|)
%    \begin{macrocode}
\DeclareRobustCommand*\subref{%
  \@ifstar
    {\caption@withoptargs\subcaption@ref*}%
    {\caption@withoptargs\@subref}}
\newcommand*\@subref[2]{%
  \caption@ifundefined\hyperref
    {\subcaption@ref{#1}{#2}}%
    {\hyperref[#2]{\subcaption@ref{*#1}{#2}}}}
%    \end{macrocode}
%    \begin{macrocode}
\newcommand*\subcaption@ref[2]{%
  \begingroup
    \caption@setoptions{sub}%
    \subcaption@reffmt\p@subref{\ref#1{sub@#2}}%
  \endgroup}
%    \end{macrocode}
%    \begin{macrocode}
\newcommand*\p@subref{}
%    \end{macrocode}
% \end{macro}
%
%    \begin{macrocode}
\DeclareCaptionOption{subrefformat}{\subcaption@setrefformat{#1}}
%    \end{macrocode}
%
% \begin{macro}{\subcaption@setrefformat}
%  |\subcaption@setrefformat|\marg{name}\par
%  Selecting a subref format simply means saving the code (in |\subcaption@reffmt|).
%    \begin{macrocode}
\newcommand*\subcaption@setrefformat[1]{%
  \@ifundefined{caption@lfmt@#1}%
    {\caption@Error{Undefined label format `#1'}}%
    {\expandafter\let\expandafter\subcaption@reffmt\csname caption@lfmt@#1\endcsname}}
%    \end{macrocode}
%    \begin{macrocode}
\subcaption@setrefformat{simple}
%    \end{macrocode}
% \end{macro}
%
% \iffalse
%</package>
% \fi
%
% \iffalse
% --------------------------------------------------------------------------- %
% \fi
%
% \Finale
%
\endinput
