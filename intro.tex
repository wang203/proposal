Social media encourages the development of online social network, connecting billions of users sharing status and comments in text and multimedia format.
These increasingly large number of multimedia files such as photographs become a new source of web-scale data with high variety and publicly available.
Along with the image files and text annotations, social networking websites and mobile applications also record user profiles as well as time and location metadata of users activity.
This new data source is quite noisy but is also easy accessible. 
It collects data constantly from a broad selection of locations on this earth. 
To better understand the information behind this large collection of data, there is a critical demand for novel, scalable methods to navigate the visual content together with the textual annotation and user activity.
Researchers from computer vision, multimedia, data mining and machine learning become more and more interested in this interdisciplinary area, especially with the progress in computer vision and with the development of computing capacity.

When we integrate the visual content by the metadata of user activities, we will provide a very low-cost, large scale data source for researchers in many areas of social and natural science.
For example, we will help research in social science when we are interested in the human behavior of those who take photos and share on social media, and in nature science when we are interested in the natural scenes and the associated location, time and user profile.
These attractive properties of this new data source also bring us new challenges. 
Being able to efficiently processing data with advanced computer vision techniques is an important requirement.
To accommodate the graphical structure of metadata in social media, we also need to develop new models and systems.
% challenge of color project

This thesis introduces our works of solving problems in three aspects to facilitate research in different areas with web-scale image data.
